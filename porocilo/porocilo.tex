\documentclass[a4paper, 12pt]{article} %%%here01

\usepackage[slovene]{babel}
\usepackage[utf8]{inputenc}
\usepackage[T1]{fontenc}
\usepackage{lmodern}
\usepackage{units}
\usepackage{eurosym}
\usepackage{amsmath}
\usepackage{amssymb}
\usepackage{amsthm}
\usepackage{amsfonts}
\usepackage{mathtools}
\usepackage{graphicx}
\usepackage{color}
\usepackage{url}
\usepackage{hyperref}
\usepackage{enumerate}
\usepackage{enumitem}
\usepackage{pifont}

\definecolor{airforceblue}{rgb}{0.36, 0.54, 0.66}
\definecolor{bostonuniversityred}{rgb}{0.8, 0.0, 0.0}

\newcommand{\Zn}{\mathbb{Z}_n}
\renewcommand{\P}{\mathbb{P}}

\newenvironment{matematika}[1]{
\textcolor{bostonuniversityred}{\underline{\textsc{#1:}}}
}{
}

\setlength{\parindent}{0mm}

\begin{document}
%naslov
\begin{titlepage}
\centering
\textbf{\Huge{Presečišča Bezierjevih krivulj}}
\vfill
\textbf{\LARGE{Pristop s subdivizijskim algoritmom}}
\vfill\vfill
\textsc{\Large{Benjamin Benčina}}
\vfill\vfill
\textsc{\large{Univerza v Ljubljani}}

\textsc{\large{Fakulteta za matematiko in fiziko}}

\textsc{\large{Oddelek za matematiko}}
\vfill\vfill\vfill
	
{\large\today}

\end{titlepage}

%kazalo
\tableofcontents
\newpage

\section{Uvod}

\section{Ovojnice}

\section{Presečišča mnogokotnikov}

\section{Subdivizija Bezierjeve krivulje}

\section{Presečišča dveh Bezierjevih krivulj}

\section{Kdaj se robota srečata}

\section{Zaključek}

\section*{Viri}

\end{document}